\documentclass[11pt]{article}

\usepackage{lmodern}
\usepackage[T1]{fontenc}

\usepackage{underscore}
\usepackage{verbatim}
\usepackage{enumitem}

\usepackage[nofancy]{latex2man}

\usepackage[margin=1.25in,letterpaper]{geometry}

\setlength{\parindent}{0pt}
\setlength{\parskip}{\smallskipamount}

\begin{document}

%%% SET THE DATE
\setDate{July 2013}
\setVersion{1.0}

\begin{Name}{3}{state_history_type}{Neil N. Carlson}{Petaca}{The state_history_type module}
%%% THE ABSTRACT GOES HERE
The \texttt{state_history_type} module provides a structure for maintaining
the recent history of a solution procedure that is characterized by a (time)
sequence of state vectors, and methods for performing polynomial interpolation
based on that history.
\end{Name}

\section{Synopsis}
\begin{description}[style=nextline]
\item[Usage]
  \texttt{use :: state_history_type}
\item[Derived Types]
  \texttt{state_history}
\end{description}

\section{The state_history derived type}
The derived type \texttt{state_history} implements a scheme for maintaining
the recent history of a solution procedure characterized by a (time) sequence
of state vectors, and provides a method for computing polynomial interpolation
based on that history.  ODE integration algorithms often require such a
capability, for example.  The sequence of state vectors is stored internally
as a table of divided differences, which is easily updated and which makes
polynomial interpolation particularly simple to express.

Object of this derived type:
\begin{itemize}\setlength{\itemsep}{0pt}
\item
  should not be used in assignment statements; only the default intrinsic
  assignment is available, and its semantics are unlikely to be what is
  desired.
\item
  are properly finalized when the object is deallocated or otherwise
  ceases to exist.
\end{itemize}

The derived type has the following type bound procedures.  The state vectors
are implemented as rank-1 real arrays, and all real arguments and function
results are of kind \texttt{real_kind()} (see the type bound function below).
Currently this is \texttt{int64} from the intrinsic \texttt{iso_fortran_env}
module.

\subsection{Type bound subroutines}
\begin{description}[style=nextline]\setlength{\itemsep}{0pt}
\item[\texttt{init(mvec, \LBr t, x \Lbr,xdot\Rbr | vlen\RBr)}]
  initializes the object to maintain up to \texttt{mvec} state vectors.
  In the first variant, the vector \texttt{x} with time index \texttt{t}
  is recorded as the initial state vector of a new history.  If the optional
  vector \texttt{xdot} is also specified, it is recorded as the state
  vector time derivative at the same time index.  It must have the same
  size as \texttt{x} and the size of \texttt{x} establishes the expected
  size of all further state vector arguments.  In the second variant,
  \texttt{vlen} specifies the length of the vectors to be maintained but
  no state vector is recorded.  An object must be initialized before any
  of the other methods are invoked.  It is permitted to re-initialize an
  object.
\item[\texttt{flush(t, x \Lbr,xdot\Rbr)}]
  flushes the accumulated state vectors and records the state vector
  \texttt{x} with time index \texttt{t} as the initial state vector of
  a new history.  If \texttt{xdot} is specified, it is also recorded as
  the state vector time derivative at the same time index.  This differs
  from \texttt{init} in that the maximum number of vectors and their
  lengths are not changed.
\item[\texttt{record_state(t, x \Lbr,xdot\Rbr)}]
  records the vector \texttt{x} with time index \texttt{t} as the most recent
  state vector in the history.  If the vector \texttt{xdot} is present, it is
  recorded as the state vector time derivative at the same time index.  The
  oldest state vector (or two oldest in the case \texttt{xdot} is present) is
  discarded once the history is fully populated with \texttt{mvec} vectors.
  Note that when only one of a \texttt{x}/\texttt{xdot} pair of vectors is
  discarded, it is effectively the derivative vector that gets discarded.
\item[\texttt{get_last_state_copy(copy)}]
  copies the last recorded state vector into the array \texttt{copy},
  whose length should equal \texttt{state_size()}.
\item[\texttt{get_last_state_view(view)}]
  associates the array pointer \texttt{view} with the last recorded state
  vector.  \emph{This should be used with great caution.}  The target of
  this pointer should never be modified or deallocated.  The pointer will
  cease to reference the last recorded state vector when the history is
  subsequently modified through calls to \texttt{record_state}, \texttt{flush},
  or \texttt{init}.
\item[\texttt{interp_state(t, x \Lbr,first\Rbr \Lbr,order\Rbr)}]
  computes the interpolated state vector at time index \texttt{t} from the
  set of state vectors maintained by the object, and returns the result in
  the user-supplied array \texttt{x}.  Polynomial interpolation is used, and
  \texttt{order}, if present, specifies the order using the \texttt{order+1}
  most recent vectors; 1 for linear interpolation, 2 for quadratic, etc.
  It is an error to request an order for which there is insufficient data.
  If not specified, the maximal order is used given the available data; once
  the history is fully populated, the interpolation order is \texttt{mvec-1}.
  Typically the array \texttt{x} would have the same size as the stored state
  vectors, but more generally \texttt{x} may return any contiguous segment of
  the interpolated state starting at index \texttt{first} (default 1) and
  length the size of \texttt{x}.
\item[\texttt{revise(index, x \Lbr,xdot\Rbr)}]
  revises the history of a selected state vector component: \texttt{index}
  is the component, \texttt{x} is the new most recent value of that component,
  and \texttt{xdot}, if present, is the new first divided difference.
  All higher-order divided differences for the component are set to zero.
  \texttt{Depth()} must be at least 1 (2 if \texttt{xdot} is present) to use
  this method.  The use-case for this method arises from equation switching.
\end{description}

\subsection{Type bound functions}
\begin{description}[style=nextline]\setlength{\itemsep}{0pt}
\item[\texttt{real_kind()}]
  returns the kind parameter value expected of all real arguments.
\item[\texttt{depth()}]
  returns the number of state vectors currently stored.  This number will
  vary between 0 and \texttt{max_depth()}..
\item[\texttt{max_depth()}]
  returns the maximum number of state vectors that can be stored.  This
  number is the value of \texttt{mvec} used to initialize the object.
\item[\texttt{state_size()}]
  returns the length of the state vectors stored by the object.  This number
  will equal the size of \texttt{x} or the value of \texttt{vlen} used to
  initialize the object.
\item[\texttt{last_time()}]
  returns the time index of the last recorded state vector.
\item[\texttt{time_deltas()}]
  returns an array containing the time index differences: the first element
  is the difference between the last and penultimate times; the second is the
  difference between the last and antepenultimate times, and so forth.  The
  length of the result equals \texttt{depth()-1}.  It is an error to call this
  method if \texttt{depth()} is less than 2.
\item[\texttt{defined()}]
  returns true if the object is well-defined; otherwise it returns false.
  Defined means that the data components of the object are properly and
  consistently defined.  The function should return true at any time after
  the \texttt{init} method has been called; it is intended to be used in
  debugging situations and is used internally in assertion checks.
\end{description}

%\section{Example}
%\begin{verbatim}
%%% SOME EXAMPLE CODE
%\end{verbatim}

\section{Bugs}
Bug reports and improvement suggestions should be directed to
\Email{neil.n.carlson@gmail.com}

\LatexManEnd

\end{document}
